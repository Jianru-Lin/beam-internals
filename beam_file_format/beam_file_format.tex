\documentclass{article}
\usepackage{footnote, fullpage, etoolbox}

% define the title
\title{BEAM File Format}
\begin{document}
\maketitle
\pagebreak
\tableofcontents
\pagebreak
\section{About this Document}

This document describes BEAM (Bogdan's Erlang Abstract Machine)
as included in the Erlang Open Telecom Platform (Erlang/OTP).\footnote{More
information on the history of BEAM and the VMs used in Erlang can be
found at http://www.erlang-factory.com/upload/presentations/247/erlang\_vm\_1.pdf}
It is intended to provide useful information on the BEAM file format
and the Virtual Machine.

More information on Erlang can be found at http://www.erlang.org/

\subsection{Status}
This is an INCOMPLETE DRAFT document,
documenting the BEAM Virtual Machine included in version R15B01 of Erlang/OTP.
This is draft 1 of this document. 

This document makes statements about what must or must not occur in a BEAM file,
and that is based on analysis of code both in erlc and the BEAM VM.
References to code are provided where possible. None of these statements
have been tested by altering BEAM byte code and checking the
results through either the BEAM VM or the BEAM disassembly capabilities of 
the Erlang language.  To provide additional proof of these assertions, 
they should be tested, and reliance on this document without those 
tests is taken at the reader's risk.  You have been warned.

\section{Introduction}
\subsection{Background}
The BEAM File format was originally based off
Electronic Arts Interchange File Format 85
(EA IFF 85).\footnote{http://www.martinreddy.net/gfx/2d/IFF.txt}
\footnote{http://www.erlang.se/\~bjorn/beam\_file\_format.html}

\subsection{Conventions}

\subsubsection{Key Words}
This document adopts the language of Internet Engineering Task Force (IETF)
Request for Comments (RFC) 2119,
``Key words for use in RFCs to Indicate Requirement
Levels''.\footnote{http://www.ietf.org/rfc/rfc2119.txt}  
This is done to ensure appropriate comprehension about absolute requirements,
situations that exist under certain conditions, and items that are optional.
There is one caveat to this adoption: While the definitions of the key words
shown there are adopted, this document does not attempt to capitalize those
key words (in the hope of providing better readability).

\subsubsection{Bit Structure}
All numbers stored in BEAM files are big-endian unless
otherwise specified.  

When a specific bit is referred to by the bit number, the 0-bit is the least
significant bit.  This will include when bits are shown as variables with an
index as a subscript.  The least-significant bit will appear on the right, unless
otherwise specified.  The example of encoding the number 10 is shown below.
\begin{tabular}{ |c|c|c|c| } \hline
X$_3$ &X$_2$ &X$_1$ &X$_0$  \\ \hline
1     &  0   &   1  &  0  \\ \hline
\end{tabular}\\

\subsubsection{Function Notation}
This document adopts the ``Erlang style'' of function
identification.\footnote{http://www.erlang.org/doc/reference\_manual/users\_guide.html
version 5.9.1, Section 5.1}  In other words, 
a function can be uniquely identified with 3 pieces of information: The module 
in which the function resides, the function name, and the arity of the function.  (The arity
is the number of arguments a function has).  We write these as: \\
\texttt{module:function/arity}

A specific example for the cosine function (abbreviated as `cos') provided in a
module called math:\\
\texttt{math:cos/1}

\subsubsection{Font Styles}
A \texttt{Fixed Width} font is used to identify short code samples or function names.



\subsection{Key Definitions}
For purposes of this document, some level of basic programming knowledge is
assumed.  No attempt is made to define common words like function, module,
or other concepts that are pervasive.  However, a number of concepts
are not as well known, but are critical to understanding BEAM files. 
These concepts are pervasive in BEAM files and VMs designed to interpret
them, such as the BEAM VM provided as part of Erlang/OTP.  Definitions
for those items are presented here.

\subsubsection{Atom}
An Atom is a non-numerical constant.  These are similar to the concept of 
enumerated types in programming languages like Java or C++.  

In BEAM files, Atoms are global.  A Virtual Machine that interprets BEAM files 
must treat Atoms as global.  Therefore, if `January' is defined as an Atom in
two BEAM files, those atoms are equal to the Virtual Machine (and would pass 
what most developers would recognize as an equality test [==]).

Note: Restrictions on Atoms may exist in programming languages that compile to
BEAM format, but since this is now a binary file and items will be specifically
identified as atoms, there is less restriction on what characters may
appear in an Atom, and there are no escape sequences in the storage of an
Atom in a BEAM file.\footnote{TODO: Check restrictions on Atoms - what characters
are not legal (if any)? Null prohibited? Others?}

\section{File Structure}
The EA IFF 85 standard defines a header for the file, followed 
by chunks.  Chunks are a byte-stream, each with a header and a
variable-length data structure (the length of which is defined in 
the chunk's header).  Each chunk must be a multiple of 4 bytes long.
The chunk must be padded to a multiple of 4 bytes if the native data
within the chunk is not a multiple of 4 bytes in length.\footnote{TODO:
Defensive test to check if the padding MUST be zeros or SHOULD be zeros?}
The first 4 bytes of the header of a chunk is a
magical value identifying the type of chunk.  This document calls that
magical value the chunk Identifier.

The data within the chunks is further structured.  Tables are a group
of zero or more fixed- or variable-sized data structures (depending
on the chunk type).  Each of the items in a Table is referred to 
in this document as a Record.  

Note: The use in this document of the `Table' and `Record' key words
may make the names of the items in the BEAM file inconsistent with
the names that appear in the Erlang Compiler (erlc)
and the BEAM Virtual Machine.  In some cases that is unavoidable (as those 
two already call the chunk by a different name), while in others, a
standardized name is adopted for this document in order to leverage
explanatory key words.

For fixed-length records, the table is written as a contiguous data 
structure.  Each record will not have an individual header.

For variable-length records, there are two side effects: First, each
record will require a header to indicate the length of the record. 
Second, it is not possible to pre-determine the location of all the
records in the table (The entire table must be read to determine the
value of the last record).

Some chunks must appear in the BEAM file, others are optional.
If a specific type of chunk appears, it must only appear one time.
Although it is unlikely one would ever reach this size, 
the maximum BEAM file size is approximately 2GB (this is due
to the form length appearing in the header as a 32-bit number of bytes).

The following chunks must appear in a BEAM file:\\
\begin{savenotes}
\begin{tabular}{ |l|l|p{3.5in}| } \hline
Identifier & Chunk Name         & Description \\ \hline
Atom       & Atom Table Chunk   & Defines Atoms used in the file \\ \hline
Code       & Code Chunk         & Defines Code to execute functions in the file \\ \hline
StrT       & String Table Chunk & Defines Strings used in the file\footnote{TODO: how is this different than LitT?} \\ \hline
ImpT       & Import Table Chunk & Defines methods that are implrted into (called from) the module \\ \hline
ExpT       & Export Table Chunk & Defines methods that are exported from the module \\ \hline
\end{tabular}\\
\end{savenotes}

The following chunks must appear in a BEAM file if they are required by references made in the Code Chunk:\\
\begin{savenotes}
\begin{tabular}{ |l|l|p{3.5in}| } \hline
Identifier & Chunk Name                 & Description \\ \hline
FunT       & Local Function Table Chunk\footnote{This is called the Lambda Chunk by the BEAM VM} & useful for cross-reference tools \\ \hline
LitT       & Literal Table Chunk        & Stores Literals (like large integers, lists, etc.) \\ \hline
Line       & Line Table Chunk           & Defines information about line numbers from the original source file \\ \hline
\end{tabular}\\
\end{savenotes}

The following chunks are optional chunks for a BEAM file.\\
\begin{savenotes}
\begin{tabular}{ |l|l|p{3.5in}| } \hline
Attr       & Attribute Chunk            & defines the attributes of the module \\ \hline
CInf       & Compile Information Chunk  & defines the compile information about the module \\ \hline
Trac       & Function Trace Chunk\footnote{TODO: Is it possible for Trac length to be zero? (currently is loaded as 4 with 4 `00' bytes when no Trac)}
                                        & If present, the loaders will insert special trace instructions to support function tracing \\ \hline
AbsT       & Abstract Code Chunk        & Contains an abstract representation of the code, useful for debugging, ignored by VM \\ \hline
LocT       & TBD                        & TBD\\ \hline
\end{tabular}\\
\end{savenotes}

Chunks in the BEAM file are not order-dependent, the loading
code of a virtual machine must ensure any loading dependencies are resolved.\footnote{This appears true from inspection, but should be tested}

\subsection{(Presumed) Design Considerations}

A number of the Tables within BEAM files serve as reference for the Code Chunk. 
This is to allow the Code Chuck to remain compact, and to avoid storing duplicate
information when a value is reused.  Each of the types of potential
reused values is stored in a different table.  For those with appropriate 
experience, these can loosely be thought of like tables in a relational database.

BEAM files contain some `hint' information for a program that is loading a BEAM file.
While these values not formally necessary to load a BEAM file (in a mathematical sense
they are duplicate information), this `hint' information is provided to help with memory
allocation.  These `hint' fields must not be ignored when BEAM files are generated, and
these fields must contain the exact value appropriate for the given field.  This will
ensure both that sufficient memory is allocated by a program loading the BEAM file and
that a program loading a BEAM file can rely on these fields for looping and validation. 

\subsection{References}

When an item defined within one of the tables is referred to by the code, this may be
called a reference.  The references are of variable length, designed to make the most
common references very small in the BEAM file, while accommodating larger values when
necessary.\footnote{See beam\_load.c \#define GetTagAndValue}

The minimum size of a reference is 1 byte.  
The 1-byte form of a reference is shown here:\\
\begin{tabular}{ |l|l|l|l|l|l|l|l| } \hline
%X$_7$ & X$_6$ & X$_5$ & X$_4$ & X$_3$ & X$_2$ & X$_1$ & X$_0$  \\ \hline
IX$_3$ & IX$_2$ & IX$_1$ & IX$_0$ &   0   & ID$_2$ & ID$_1$ & ID$_0$  \\ \hline
\end{tabular}\\

In this case, the 3-bit value of Identifier ID is 0 $<$ ID $<$ 6. 
For ID == 7, the record is variable length (see below).  
The value IX is used as the index within the appropriate 
namespace of the identifier.\footnote{TODO:
I beleive it would be beneficial to have an example of using a reference/index in a
lookup of a table.  Perhaps this is part of a larger example at some point?}

The 2-byte form of a reference is shown here:\\
\begin{tabular}{ |l|l|l|l|l|l|l|l| } \hline
%X$_7$ & X$_6$ & X$_5$ & X$_4$ & X$_3$ & X$_2$ & X$_1$ & X$_0$  \\ \hline
IX$_{10}$ & IX$_9$ & IX$_8$ &   0    &   1    & ID$_2$ & ID$_1$ & ID$_0$  \\ \hline
IX$_7$   & IX$_6$ & IX$_5$ & IX$_4$ & IX$_3$ & IX$_2$ & IX$_1$ & IX$_0$  \\ \hline
\end{tabular}\\

In this case, the 3-bit value of Identifier ID is 0 $<$ ID $<$ 6. 
For ID == 7, the record is variable length (see below).
The value IX is used as the index within the appropriate 
namespace of the identifier.

Note that the maximum value of any reference index in a BEAM file is 2047.
This is not meant to imply that all reference types are valid up to 2047.  
Some may be more limited (for example, each process has 1024
X registers\footnote{http://dl.dropbox.com/u/4764922/beam.pdf})

For ID == 7, the record header (the first byte) is:\\
\begin{tabular}{ |l|l|l|l|l|l|l|l| } \hline
%X$_7$ & X$_6$ & X$_5$ & X$_4$ & X$_3$ & X$_2$ & X$_1$ & X$_0$  \\ \hline
L$_2$ & L$_1$ & L$_0$ & x &   x   & 1 & 1 & 1 \\ \hline
\end{tabular}\\
Where L is the length (in bytes) of the rest of the record.\footnote{TODO:
Need to decode the rest of the `extended' ID == 7 format.}


\subsubsection{Reference Namespace Identifiers}

\begin{savenotes}
\begin{tabular}{ |l|p{1in}|l|p{3in}| } \hline
Identifier & Type & Assembler ID & Description \\ \hline
00 & Literal\footnote{TODO: Need to check what index Literal Table uses (is 00 a special value? - doesn't look like it from inspection, but should make an explicit test)} & u & Index to a record appearing in the Literal Chunk\\ \hline
01 & Integer               & i & An integer\\ \hline
02 & Atom                  & a & Index to a record appearing in the Atom Chunk\\ \hline
03 & X Register            & x & Index to an X Register in the Virtual Machine\\ \hline
04 & Y Register            & y & Index to a Y Register in the Virtual Machine\\ \hline
05 & Label (i.e. Function) & f & Index to a Label in Code Chunk\\ \hline
06 & Character\footnote{Cannot be generated by the current version of erlc.  Not yet tested if BEAM VM still supports use of this Identifier}  & h & \\ \hline
07 & Extended Format       & z & (List, Float, etc.)\\ \hline
\end{tabular}\\
\end{savenotes}

Atom 0 is the special atom NIL\footnote{TODO: Find reference for this information
- it needs it and I didn't write it down in the first pass}

X and Y are zero-indexed (there IS an x,0 and a y,0 register)

Label is 1 indexed for the creator of a BEAM file, the 0 value is special:
it contains the undefined location\footnote{This is in beam\_load.c}

Character must be a value from 0$<$c$<$65535 (must be a valid ISO 8851-1 character).

\subsection{Form Header}
The Form Header is a 12-byte section at the beginning of the file.
This is the structure of the Form Header:\\
\begin{tabular}{ |l|l|p{3in}| } \hline
Length  & Value  & Description\\ \hline
4 bytes & `FOR1' & Magic number indicating the IFF form.  This is an extention to IFF indicating that all chunks are four-byte aligned.\\ \hline
4 bytes & length & Form length (file length in bytes - 8)\\ \hline
4 bytes & `BEAM' & Form type (Magic number for BEAM files)\\ \hline
\end{tabular}\\

The form header is followed by the chunks.

\subsection{The Atom Chunk}
The Atom Chunk stores Atoms - the constant vales
(including identifiers like the module name and function names). 
The Atom Chunk is composed of a Header followed
by Atom definitions.

\subsubsection{Header}
The Atom Chunk Header is composed of 12 bytes.
This is the structure of the Atom Chunk Header:\\
\begin{tabular}{ |l|l|p{3in}| } \hline
Length  & Value  & Description\\ \hline
4 bytes & `Atom' & Magic number indicating the Atom Chunk.\\ \hline
4 bytes & size & Atom Chunk length in bytes\\ \hline
4 bytes & count & Number of atoms in the Atom Chunk\\ \hline
\end{tabular}\\

\subsubsection{Atom Definition Format}
Each Atom Definition is composed of two items:
This is the structure of the Atom Definition:\\
\begin{tabular}{ |l|l|p{3in}| } \hline
Length  & Value  & Description\\ \hline
1 bytes & length & Number of bytes in the Atom\\ \hline
n bytes & atom & The Atom (in the form of ASCII characters). \\ \hline
\end{tabular}\\

Restriction: Atoms are limited to 255 characters or less.  Characters must be valid characters in ISO 8859-1.

%Tip: In Erlang, Unquoted Atoms start with lowercase letters followed by a sequence of alphanumeric characters, the underscore \_ or the at sign (@). - but ANY atom can be created with a single quoted version of the atom, e.g. 'January'.
%Tip: If this is violated, erlc will give you the error: "illegal atom"

\subsubsection{Requirements}
The first Atom to appear in the Atom Chunk must be the module name.

\subsection{The Code Chunk}
The Code Chunk is a mandatory chunk that stores the code for the module.  
The Code Chunk is composed of a header followed by one or more function definitions.\footnote{TODO:
Need to check whether the module\_info items are required in BEAM files?  By indications so
far, they are not required as they are just delegations to functions in the erlang module}
\footnote{TODO: Check if code must to ``belong to'' a function.  While elc will not produce it,
can a well structured BEAM file
start with code that is not part of a function or will the VM loader error on it? 
If that code is not preceded by a label, then it will produce code that is unreachable. 
Is that illegal or just wasteful?}

\subsubsection{Header}
The Code Chunk Header is composed of 28 or more bytes.
This is the structure of the Code Chunk Header:\\
\begin{savenotes}
\begin{tabular}{ |l|l|p{3in}| } \hline
Length  & Value  & Description\\ \hline
4 bytes & `Code' & Magic number indicating the Code Chunk.\\ \hline
4 bytes & size & Code Chunk length in bytes\\ \hline
4 bytes & info-size\footnote{TODO: Need to understand purpose of information fields, and what a loader should do if it encounters information fields that it does not understand - are information fields supposed to be required or are they optional information, or ??  Need to establish this in order to make MUST vs SHOULD statements}
                     & Length of the information fields before code.  This is for future expansion.  It must be set to zero when the instruction set identifier is 0.\\ \hline
4 bytes & version & Instruction set version.  This is 0 for OTP R5--R15.  This will be incremented if the instructions are changed in incompatible ways (instructions renumbered, argument types changed, etc.)\\ \hline
4 bytes & opcode-max & The highest opcode used in the code section. This allows addition of new opcodes without incrementing the version of the BEAM file.\\ \hline
4 bytes & labels\footnote{TODO: As defensive checking, should check whether labels can go unused, and if a loader would complain in any way.  In that way we can define if MUST use all labels implied by the labels field or SHOULD not leave labels unused}
                  & The number of labels.  This is a hint for the loader to help allocate the label table.\\ \hline
4 bytes & function-count & The number of functions contained in the Code Chunk.\\ \hline
\end{tabular}\\
\end{savenotes}

Note: The value in the labels field holding the number of labels must be 
1 greater than the number of labels in the BEAM file (remembering that 
labels are indexed from zero).  This is because the 0 label is reserved 
for the undefined location.  The compiler must set the labels value equal 
to the highest label it uses + 1.

\subsubsection{Code Definition Format}
The code is simply defined as a block of code.  Special instructions 
identify where functions insert 
into the code.\footnote{TODO: This is about behavior of the opcodes, 
but important to answer for code structure in BEAM:
Defensive test to see if it's possible to have fallthrough in such a 
way that a function does not return
but rather passes through a label or passes through a func\_info.
(e.g. are the function\_info and label calls definitional only or 
does it actually insert some instructions into the stream?)}

Each operation is coded according to the opcodes defined in the (not yet existing) Opcode section


\subsubsection{Requirements}
A BEAM file loader should not load a BEAM file if:
(a) It does not understand the instruction set version of the BEAM file
(b) The opcode-max is higher than the greatest opcode that the loader comprehends

\subsection{String Table Chunk}
A String Table Chunk is used when there are string values to be referenced in the BEAM
file.\footnote{TODO: Is this still used?  Can it be written?  Would like example code
that produces a non-null String Table Chunk.}

\subsubsection{Header}
The String Chunk Header is composed of 8 bytes.
This is the structure of the String Chunk  Header:\\
\begin{tabular}{ |l|l|p{3in}| } \hline
Length  & Value  & Description\\ \hline
4 bytes & `StrT' & Magic number indicating the String Table Chunk.\\ \hline
4 bytes & size & String Table Chunk length in bytes\\ \hline
\end{tabular}\\

\subsubsection{String Definition Format}

TODO: Format to be validated after a code example is found.

\subsection{Import Table Chunk}
The Import Table Chunk is used to define methods in other modules
that are called from the current module.

\subsubsection{Header}
The Import Table Chunk Header is composed of 8 bytes.
This is the structure of the Import Table Chunk Header:\\
\begin{tabular}{ |l|l|p{3in}| } \hline
Length  & Value  & Description\\ \hline
4 bytes & `ImpT' & Magic number indicating the Import Table Chunk.\\ \hline
4 bytes & count & Import Table Chunk length in the number of records \\ \hline
\end{tabular}\\

Note each record is 12 bytes.  The size of the data for the Import Table Chunk 
will be equal to 12 bytes * number of records

\subsubsection{Import Record Format}
The Import Record Format is a fixed format of 12 bytes per record.
This is the structure of the Import Record Format:\\
\begin{tabular}{ |l|l|p{3in}| } \hline
Length  & Value  & Description\\ \hline
4 bytes & module-atom-id & An identifier of the atom used to identify the module from which the method is imported.  The identifier is used to look up the Atom in the Atom Chunk of this BEAM file.\\ \hline
4 bytes & method-atom-id & An identifier of the atom used to identify the method name of the method to be imported.  The identifier is used to look up the Atom in the Atom Chunk of this BEAM file.\\ \hline
4 bytes & arity & The arity of the method to be imported\\ \hline
\end{tabular}\\

Together these 3 items contain the information we would normally associate with a method signature, the module, function name, and arity, such as:
\texttt{math:cos/1} where\\
\texttt{math} is the module\\
\texttt{cos} is the function name\\
\texttt{1} is the arity.\\

\subsection{Export Table Chunk}
The Export Table Chunk is used to define methods that this module exports, and
therefore those methods which can be called by other modules.

\subsubsection{Header}
The Export Table Chunk Header is composed of 8 bytes.
This is the structure of the Export Table Chunk Header:\\
\begin{tabular}{ |l|l|p{3in}| } \hline
Length  & Value  & Description\\ \hline
4 bytes & `ExpT' & Magic number indicating the Export Table Chunk.\\ \hline
4 bytes & count & Export Table Chunk length in the number of records\\ \hline
\end{tabular}\\

Note each record is 12 bytes.  The size of the data for the Export Table Chunk will be equal to 12 bytes * number of records

\subsubsection{Export Record Format}
The Export Record Format is a fixed format of 12 bytes per record.
This is the structure of the Export Record Format:\\
\begin{tabular}{ |l|l|p{3in}| } \hline
Length  & Value  & Description\\ \hline
4 bytes & method-atom-id & An identifier of the atom used to identify the method name of the method to be exported.  The identifier is used to look up the Atom in the Atom Chunk of this BEAM file.  Note that for purposes of this index, the list of Atoms in the Atom Chunk is an index *starting at one*.\\ \hline
4 bytes & arity & The arity of the method to be exported\\ \hline
4 bytes & module-label & The code label identifing the starting position of the method within the Code Chunk of the BEAM file.\\ \hline
\end{tabular}\\

Together these 3 items contain the information required to being execution at 
the appropriate point. We need to sets of information to begin execution of a 
method.  We need the method signature we are familiar with:
\texttt{math:cos/1} where\\
\texttt{math} is the module\\
\texttt{cos} is the function name\\
\texttt{1} is the arity.\\
...and the location (label) in the code block to begin execution. 
The module name is not required in the export table, since the module name
is known from the first entry in the Atom Chunk.  It is shared among all of 
the exported methods.  The other required information is contained within the 
Export Record.

\subsection{Literal Table Chunk}
The Literal Table Chunk is used to define literals used in the module.  
These literals are stored in a compressed state in the BEAM file.

\subsubsection{Header}
The Literal Table Chunk Header is composed of 8 bytes.
This is the structure of the Literal Table Chunk Header:\\
\begin{tabular}{ |l|l|p{3in}| } \hline
Length  & Value  & Description\\ \hline
4 bytes & `LitT' & Magic number indicating the Literal Table Chunk.\\ \hline
4 bytes & bytes & Compressed Literal Table length in bytes\\ \hline
\end{tabular}\\

Note the bytes provided above is the bytes of the compressed record.

\subsubsection{Compressed Literal Table Format}
The Literal Table Chunk contains a single Compressed Literal Table.  This 
Compressed Literal Table contains a 4-byte header followed by the compressed
data.  The header is a hint to the loader that indicates the uncompressed size of the 
compressed data in the Compressed Literal Table.
This is the structure of the Compressed Literal Table Header:\\
\begin{tabular}{ |l|l|p{3in}| } \hline
Length  & Value  & Description\\ \hline
4 bytes & bytes & Uncompressed Literal Table Length in bytes\\ \hline
\end{tabular}\\

The compression is done with zlib.  Once uncompressed, the result is the Uncompressed
Literal Table.

\subsubsection{Uncompressed Literal Table Format}
The Uncompressed Literal Table contains one or more literal records.
It contains a 4-byte header followed by the records.
The header indicates the number of literal records contained in the Uncompressed Literal Table.
This is the structure of the Uncompressed Literal Table Header:\\
\begin{tabular}{ |l|l|p{3in}| } \hline
Length  & Value  & Description\\ \hline
4 bytes & records & Number of Literal Records in the Uncompressed Literal Table\\ \hline
\end{tabular}\\

\subsubsection{Literal Record Format}
Each record in the Uncompressed Literal Table is composed of a header and data.
This is the structure of the Literal Record Format:\\
\begin{tabular}{ |l|l|p{3in}| } \hline
Length  & Value  & Description\\ \hline
4 bytes & n & Number of bytes in the Literal\\ \hline
n bytes & data & Data used to store the Literal.  This is in external format\\ \hline
\end{tabular}\\

\subsection{Line Table Chunk}
A Line Table Chunk is used to help trace line numbers within the original code file.

\subsubsection{Header}
The Line Table Chunk Header is composed of 8 bytes.
This is the structure of the Line Table Chunk Header:\\
\begin{tabular}{ |l|l|p{3in}| } \hline
Length  & Value  & Description\\ \hline
4 bytes & `Line' & Magic number indicating the Line Table Chunk.\\ \hline
4 bytes & size & Line Table Chunk length in bytes\\ \hline
\end{tabular}\\

\subsubsection{Line Table}
The Line Table has a 20-byte header.  The data stored in the header is:
This is the structure of the Line Table Header:\\
\begin{savenotes}
\begin{tabular}{ |l|l|p{3in}| } \hline
Length  & Value  & Description\\ \hline
4 bytes & version & Line Table Version (curently must be set to zero)\\ \hline
4 bytes & flags & Reserved for future use = ignore\\ \hline
4 bytes & count & Number of Line instructions in the Code Chunk\\ \hline
4 bytes & records & Number of Line Records in the Line Table (the data that follows the Line Table Header)\\ \hline
4 bytes & fnames\footnote{TODO: Establish how File Name Records would be created... since they are not present in default compilation by erlc (at least for stuff I've written/analyzed)} & Number of File Name Records in the Line Table\\ \hline
\end{tabular}\\
\end{savenotes}

The Line Table header is followed by the Line Records.  Each record is coded 
in the Argument Encoding format.

The Line Records are followed by File Name Records.  Each record is coded 
as a 16-bit integer length, followed by the number of bytes provided in the 
length (presumably of ISO 8859-1 characters).

Note: The Line items table is actually one-indexed for user use, the 0th
item in that table is special and is used to store the undefined
location.\footnote{TODO: Check if this is BEAM VM specific or if this is
intended to be a generic characteristic of BEAM files - not quite sure
yet how to call this one.}



\section{External Format}

The external format, used in the Literal Table Chunk, is documented at\\
http://www.erlang.org/doc/apps/erts/erl\_ext\_dist.html



\section{Key Acronyms}

BIF: Built-in functions.  These are functions built into the vitual machine.  These
are effectively part of Erlang, and are (mostly) considered part of the erlang module.

\section{Potential Work Items}

The definition of Tables in BEAM files are overloaded.  This is because both fixed-length 
records (some might call these arrays) and variable length records (what might be best called
tables) are called Tables.  This is not necessarily a problem, and calling ExpT the Export
Array might be more trouble than it's worth.  So the key question is a very specific language
syntax question: Do we want to define fixed-length things as arrays to try to help clarify
fixed vs. variable length?




\end{document}

